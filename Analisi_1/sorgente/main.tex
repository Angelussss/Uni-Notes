\documentclass{article}
\usepackage{graphicx} % Required for inserting images
\usepackage[utf8]{inputenc}
\usepackage[T1]{fontenc}
\usepackage{amsmath,amsthm}
\usepackage[margin=3cm]{geometry} % Tutti i margini a 2 cm
\usepackage{dsfont}
% Definiamo uno "stile" di teorema personalizzato:
\newtheoremstyle{mystyle}% Nome dello stile
  {}        % Spazio sopra
  {}        % Spazio sotto
  {}        % Font del corpo
  {}        % Indent
  {\bfseries} % Font dell'intestazione (titolo)
  {}        % Punteggiatura dopo il titolo
  {\newline}% Spazio dopo il titolo
  {%
    % #1 = nome ambiente
    % #2 = numero (se presente)
    % #3 = note (argomento opzionale fra [])
    % #4 = separatore
    \thmnote{#3.}\quad
  }

% Applichiamo lo stile a un ambiente SENZA numero.
% Il secondo argomento "" fa sì che non compaia "Definizione" o altro prefisso.
\theoremstyle{mystyle}
\newtheorem*{mydefinition}{}


\title{Analisi I}
\author{Angelo Perotti}


\begin{document}
\maketitle
\newpage
\tableofcontents
\newpage

\section{Introduction}

\section{Nozioni Preliminari}

\subsection{Insiemi}

\subsubsection{insieme matematico}
\begin{mydefinition}[Insieme matematico]
Un \emph{insieme matematico} è una collezione di oggetti
(o elementi) ben definiti, considerati nel loro insieme
come un’entità unica. \newline es:
\[
   \{0, 1, 2, 3, 4, 5, 6, 7, 8, 9\}
\]
\[
   \{a, b, c, d, e, f, g, h\}
\]
\newline
classificazione:
\begin{itemize}
    \item per \emph{elencazione}
        \begin{itemize}
            \item L'ordine degli elementi non e' importante
            \item $\in$ = appartiene, $\notin$ = non appartiene
            \item ":=" = e' definito, "$ \{ \} $" = definiscono un insieme
        \end{itemize}
    \item per \emph{proprieta' che li accomuna}
\end{itemize}
\end{mydefinition}
\newline
\subsubsection{operazioni fondamentali tra insiemi}
\begin{itemize}
    \item \textbf{unione} $$ A\cup B := \{ x:x \in A \; o \; x \in B \}$$
    "A unito B" e i suoi elementi sono dati dagli elementi di A con gli elementi di B
    \item \textbf{intersezione}$$ A\cap B := \{ x:x \in A \; e \; x \in B \}$$
    "A intersecato B" e i suoi elementi sono dati dagli elementi comuni di A e B
    \item \textbf{differenza insiemistica}$$ A/B := \{ x:x \in A \; e \; x \notin B \}$$
    "A meno B" e i suoi elementi sono gli elementi di A che non sono in B
\end{itemize}

$\hookrightarrow$ affinche' le ultime due operazioni abbiano senso introduciamo l'insieme vuoto: $\emptyset$ 

\subsection{ I numeri reali}
\begin{mydefinition}[insiemi numerici] 
insieme:
\newline
    \begin{itemize}
        \item  dei \emph{numeri naturali} $\mathds{N}=\{ 0,1,2,3,4... \}$
        \item  dei \emph{numeri interi} $\mathds{Z}=\{ 0,1,-1,2,-2... \}$
        \begin{itemize}
            \item insieme simmetrico
            \item e' chiuso rispetto la sottrazione
        \end{itemize}
        \item  dei \emph{numeri naturali} $\mathds{Q}= \bigl \{ \dfrac{p}{q}:p,q\in \mathds{Z} \hspace{3mm} q\neq 0 \bigr \}$
        \begin{itemize}
            \item  chiuso rispetto le operazioni elementari
        \end{itemize}
        
    \end{itemize}
\end{mydefinition}

\begin{mydefinition}[Teorema $\sqrt{2}$]
    non esiste alcun numero razionale   $x \in \mathds{Q}$ t.c. $x^2=2$
    \newline
    \emph{Dimostrazione}: \newline
    non ho voglia di farla ora :D
    \emph{Osservazione}: 
    \newline
    Dal teeorema deduciamo che per esempio $\sqrt{2} \in \mathds{Q}$ quindi i numeri razionali NON bastano a contenere tutte le espressioni numeriche
\end{mydefinition}

\begin{mydefinition}[Rappresentazione decimale]
ogni numero razionale $x \in \mathds{Q}$ si puo' scrivere con un allineamento decimale limitato o periodico
$$  x=\pm p \; \alpha_1 \; \alpha_2 \; \alpha_3 \; \alpha_4... \;\alpha_n... \hspace{3mm} con \hspace{3mm} p \in \mathds{N} \;e \;\alpha_i \in {0,1,2,3,4...}
$$
  
\end{mydefinition}

\begin{mydefinition}[i numeri reali]
    Definiamo l'insieme $\mathds{R}$ dei numeri reali come l'insieme di tutti i possibili allineamenti decimali
    $$\mathds{N}\subset\mathds{Z}\subset\mathds{Q}\subset\mathds{R}$$
\end{mydefinition}

 \begin{mydefinition}[intervalli]
     dati due numeri reali a,b $\in\mathbf{R}$ con a<b, si pone: \newline
     \textit{intervalli limitati}
     \begin{itemize} 
         \item (a,b) \hspace{3cm} intervallo \emph{aperto}
         \item $[$a,b$]$ \hspace{3cm} intervallo \emph{chiuso}
         \item (a,b$]$ \hspace{3cm} intervallo \emph{aperto in a, chiuso in b}
         \item $[$a,b) \hspace{3cm} intervallo \emph{chiuso in a, aperto in b}
     \end{itemize}
 \textit{intervalli illimitati}
     \begin{itemize} 
         \item (-$\infty$,a) \hspace{3cm} $\left \{ x\in \mathbf{R}, \; x<a\right \}$
         \item (-$\infty$,a$]$ \hspace{3cm} $\left \{ x\in \mathbf{R}, \; x\leq a\right \}$
         \item (a,+$\infty$) \hspace{2.87cm} $\left \{ x\in \mathbf{R}, \; x>a\right \}$
         \item $[$a,+$\infty$) \hspace{3cm} $\left \{ x\in \mathbf{R}, \; x\geq a\right \}$
     \end{itemize}
 
 \end{mydefinition}
\vspace{1cm}
 \begin{mydefinition}[Maggioranti/Minoranti]
     Sia $A\subseteq \mathds{R}$, $A\neq \emptyset$ \newline Def: \begin{itemize}
         \item Maggiorante: un elemento M $\in \mathds{R}$ si dice maggiorante di A se $$x \leq M \hspace{0.5 cm} \forall x \in A$$
         \item Minorante: un elemento m $\in \mathds{R}$ si dice minorante di A se $$x \geq m \hspace{0.5 cm} \forall x \in A$$
     \end{itemize}
     \vspace{0.5cm}
     Esistono insiemi privi di maggioranti e/o minoranti
     \begin{itemize}
         \item A si dice \emph{limitato superiormente} se ammette almeno un maggiorante
         \item A si dice \emph{limitato inferiormente} se ammette almeno un minorante
         \item A si dice \emph{limitato} se e' limitato sia superiormente che inferiormente
     \end{itemize}
     \newline
     Es: \newline
     A = $[$ 1, $\infty)$
     \begin{itemize}
         \item \textbf{A non ha maggioranti}; infatti se esistesse un maggiorante chiamato M $\in \mathds{R}$, allora dalla definizione di maggiorante deduciamo che $\forall x \in [1,+ \infty)$ si ha che x $\leq$M. Ma questo e' assurdo perche' per esempio M+1$\in[1,+\infty$), questo pero' non verifica x$\leq$M!
         \item A ammmette minoranti, per esempio m=1 oppure ogni reale minore di 1
     \end{itemize}
     A quindi: \begin{itemize}
         \item non e' superiormente limitato
         \item e' inferiormente limitato
         \item non e' limitato
     \end{itemize}

     \emph{NB} \begin{itemize}
         \item M$\in \mathbf{R}$ e' maggiorante di A se x $\leq$ M $\forall x \in A$
         \item Nella definizione di maggiorante/minorante di un insieme A \emph{NON e' richiesto} che il maggiorante/minorante \emph{appartenga} ad A
     \end{itemize}
 \end{mydefinition}

 \begin{mydefinition}[Massimo/minimo]
      Sia $A\subseteq \mathds{R}$, $A\neq \emptyset$ \newline Def: \begin{itemize}
          \item Un elemento M $\in \mathbf{R}$ si dice massimo di A se: \begin{itemize}
              \item M e' maggiorante di A
              \item M $\in$ A
          \end{itemize}
           \item Un elemento m $\in \mathbf{R}$ si dice minimo di A se: \begin{itemize}
              \item m e' minorante di A
              \item m $\in$ A
          \end{itemize}
      \end{itemize}
     \emph{NB}: \begin{itemize}
         \item Se un insieme e' limitato \textbf{superiormente}/\textit{inferiormente}, il \textbf{massimo}/\textit{minimo} puo' non esistere
         \item Massimo e minimo se esistono sono unici
     \end{itemize}
     Es: \newline
     Dato $A \subseteq \mathbf{R}, A\neq 0, A=(1,+\infty)$
     \begin{itemize}
         \item non ammette massimo (non ha maggiorante)
         \item ammette minoranti
         \item non ammette minimi, infatti l'insieme dei minoranti B=(- $\infty$, 1$]$ poiche' A$\cap$B=$\emptyset$ allora A non ammette minimo
     \end{itemize}
 \end{mydefinition}

 \begin{mydefinition}[Estremo superiore/inferiore]
          Sia $A\subseteq \mathds{R}$, $A\neq \emptyset$ \newline Def: \begin{itemize}
              \item un elemento $\overline{x} \in \mathbf{R}$ si dice \emph{estremo superiore} di A ($\overline{x}=supA)$ se $\overline{x}$ e' il \emph{piu' piccolo dei maggioranti} $$supA=min \left \{ M \in \mathbf{R}:M \hspace{1.5mm} maggiorante \hspace{1.5mm} di \hspace{1.5mm} A\right \}$$
              
              \item un elemento $\overline{x} \in \mathbf{R}$ si dice \emph{estremo inferiore} di A ($\overline{x}=infA)$ se $\overline{x}$ e' il \emph{piu' grande dei minoranti} $$infA=max \left \{ m \in \mathbf{R}:m \hspace{1.5mm} minorante \hspace{1.5mm} di \hspace{1.5mm} A\right \}$$
          \end{itemize}
          \emph{NB}: \begin{itemize}
              \item se A non e' superiormente limitato => supA=$+\infty$
              \item se A non e' inferiormente limitato => infA=$-\infty$
              \item sia A$\subset \mathbf{R}$, A $\neq \emptyset$ e supponiamo che A ammetta massimo/minimo, allora questo e' unico
              \item sia A$\subset \mathbf{R}$, A $\neq \emptyset$ e supponiamo che A ammetta \textbf{massimo M}/\textit{minimo m}, allora vale che \textbf{supA=M}/\textit{infA=m}
          \end{itemize}
 \end{mydefinition}
 \begin{mydefinition}[Assioma di Completezza di $\mathds{R}$]
     
 \end{mydefinition}
 \begin{mydefinition}[parte intera]
     
 \end{mydefinition}
 \begin{mydefinition}[Proprieta' di Archimede]
     L'insieme dei numeri naturali $\mathds{N}$ non e' superiormente limitato \hspace{2cm} $sup\mathds{N}=+\infty$
     $$\forall x \in \mathds{R} \hspace{2.5mm} \exists n\in \mathds{N} \hspace{5mm} t.c. \hspace{5mm}n>x$$
 \end{mydefinition}
 \begin{mydefinition}[Densita' di $\mathds{Q}$ in $\mathds{R}$]
     
 \end{mydefinition}
\subsection{Funzioni}
\begin{mydefinition}[Funzione]
    Siano A,B due insiemi non vuoti. Una funzione $f$ da A a B e' una corrispondenza che associa ad ogni elemento $x \in A$ uno ed un solo elemento $y \in B$
    $$\forall\hspace{1mm} x \hspace{1mm}\in\hspace{1mm} A\hspace{1mm} \exists ! \hspace{1mm}y \hspace{1mm}\in\hspace{1mm} B : \hspace{1mm}y=f(x)$$
    Notazione: $f:A\xrightarrow{}B$
    \begin{itemize}
        \item A Dominio di $f$
        \item B Codominio di $f$
        \item l'immagine di $f$ e' l'insieme dei valori della funzione
    \end{itemize}
\end{mydefinition}
\begin{mydefinition}[funzione iniettiva]
    
\end{mydefinition}
\begin{mydefinition}funzione suriettiva
    
\end{mydefinition}
\begin{mydefinition}[funzione biiettiva]
    
\end{mydefinition}
\begin{mydefinition}[composizione di due funzioni]
    
\end{mydefinition}
\begin{mydefinition}[funzione inversa]
    
\end{mydefinition}
\begin{mydefinition}[restrizione di una funzione]
    
\end{mydefinition}

\subsection{funioni reali di variabili reali}
\begin{mydefinition}[funzioni limitate]
    \begin{itemize}
        \item limitata superiormente
        \item limitata inferiormente
        \item limitata
    \end{itemize}
\end{mydefinition}
\begin{mydefinition}[funzioni simmetriche]
\begin{itemize}
    \item $f$ e' \emph{pari}
    \item $f$ e' \emph{dispari}
    \item $f$ e' \emph{periodica}
\end{itemize}
    
\end{mydefinition}
\begin{mydefinition}[funzioni monotone]
\begin{itemize}
    \item \emph{monotone crescente}
    \item \emph{monotone strettamente crescente}
    \item \emph{monotone decrescente}
    \item \emph{monotone strettamente decrescente}
\end{itemize}
    
\end{mydefinition}
\begin{mydefinition}[grafico della funzione inversa]
    
\end{mydefinition}
\subsection{le funzioni elementari e i loro grafici}
\begin{itemize}
    \item funzioni il cui grafico e' una \emph{retta}
    \begin{itemize}
        \item funzioni costanti
        \item funzioni lineari
        \item funzioni affini
    \end{itemize}
\end{itemize}

\begin{mydefinition}[valore assoluto]
    
\end{mydefinition}

\begin{mydefinition}[potenze e radici]
    
\end{mydefinition}

\begin{mydefinition}[funzioni esponenziali e logaritmiche]
    
\end{mydefinition}

\begin{mydefinition}[funzioni trigonometriche]
    
\end{mydefinition}

\begin{mydefinition}[funzioni trigonometriche inverse]
    
\end{mydefinition}

\begin{mydefinition}[funzioni iperboliche]
    
\end{mydefinition}

\section{I Numeri Complessi}
\begin{mydefinition}[forma cartesiana dei numeri complessi]
    
\end{mydefinition}
\subsubsection{operazioni sui numeri complessi}
\begin{itemize}
    \item somma
    \item Moltiplicazione
    \item Prodotto
\end{itemize}

\begin{mydefinition}[modulo e coniugato di un numero complesso]
    
\end{mydefinition}

\begin{mydefinition}[reciproco di un numero complesso]
    
\end{mydefinition}

\begin{mydefinition}[risoluzione di equazioni in $\mathds{C}$]
    
\end{mydefinition}
\subsubsection{forma trigonometrica}

\begin{mydefinition}[coordinate polari]
        
\end{mydefinition}

\begin{mydefinition}[prodotto dei numeri complessi in forma Trigonometrica]
    
\end{mydefinition}

\begin{mydefinition}[quozionte di numeri complessi in forma trigonometrica]
    
\end{mydefinition}

\begin{mydefinition}[formula di De Moivre]
    
\end{mydefinition}
\subsubsection{forme esponenziali}


\begin{mydefinition}[teorema fondamentale dell'algebra]
    un'equazione di grado n $\leq$1 in $\mathds{C}$ $$a_nz^n+a_{n-1}z^{n-1}+...+a_2z^2+a_1z^1+a_0z^0=0$$
    (dove $a_n \in \mathds{C, \hspace{0.5cm} a_n \neq 0}$) ha esattamente n soluzioni in $\mathds{C}$
\end{mydefinition}

\begin{mydefinition}[Radici]
    
\end{mydefinition}
\section{Successioni Numeriche}
\begin{mydefinition}[il fattoriale]
    
\end{mydefinition}
\begin{mydefinition}[coefficiente Binomiale]
    
\end{mydefinition}
\begin{mydefinition}[triangolo di tartaglia]
    
\end{mydefinition}
\begin{mydefinition}[il binomio di Newton]
    
\end{mydefinition}
\begin{mydefinition}[successioni geometriche]
    
\end{mydefinition}
\subsection{operazioni di limite}

\subsubsection{a$_n \xrightarrow{} 0$}
\subsubsection{a$_n \xrightarrow{} +\infty$}
\subsubsection{a$_n \xrightarrow{} -\infty$}
\subsubsection{a$_n \xrightarrow{} l$}
\subsection{limiti fondamentali}
\begin{mydefinition}[Unicita' del limite]
    
\end{mydefinition}
\begin{mydefinition}[limitezza delle successioni convergenti]
    
\end{mydefinition}

\begin{mydefinition}[sottosuccessioni e loro limiti, non esistenza del limite]
    
\end{mydefinition}
\begin{mydefinition}[limiti sottosuccessioni]
    
\end{mydefinition}
\begin{mydefinition}[esistenza del limite per successioni monotone: il numero di nepero]
    
\end{mydefinition}
\subsubsection{il numero di nepero}
\begin{mydefinition}[il numero di nepero]
    
\end{mydefinition}
\begin{mydefinition}[limite nepero generalizzato]
    
\end{mydefinition}

\begin{mydefinition}[teorema di permanenza del segno]
    
\end{mydefinition}
\begin{mydefinition}[teorema dei due carabinieri]
    
\end{mydefinition}
\subsubsection{algebra dei limiti}
\begin{mydefinition}[algebra dei limiti]
    
\end{mydefinition}
\begin{mydefinition}[forme indeterminate e algebra estesa]
    
\end{mydefinition}
\begin{mydefinition}[tecniche calcolo dei limiti]
\begin{itemize}
a seguito una lista delle piu' basilari tecniche di risoluzione dei limiti
    \item somma e sottrai
    \item raccogli chi comanda
    \item moltiplica e dividi
    \item moltiplica, dividi e razionalizza
\end{itemize}

\end{mydefinition}

\begin{mydefinition}[Criterio del rapporto]
    
\end{mydefinition}
\begin{mydefinition}[gerarchia degli infiniti]
per ogni $\alpha > 0$  e  $a>1$, le seguenti successioni rappresentano degli \emph{infiniti}, ovvero tendono a $+ \infty$ per n->$+\infty$
    $$log_an\hspace{7mm}n^\alpha\hspace{7mm}a^n\hspace{7mm}n!\hspace{7mm}n^n$$
    in ordine crescente:
    %this is a placeholder
\end{mydefinition}
\subsection{Limiti di funzioni}
\subsubsection{Intorno}
\subsubsection{Punto di Accumulazione}
\subsubsection{Limite finito di un punto}
\subsubsection{funzioni continue}
\subsubsection{limiti destro e sinistro}
\subsubsection{continuita' da destra e da sinistra}
\subsubsection{il legame con limiti di successioni}
\subsubsection{criterio per la non esistenza di un limite}
\subsubsection{teorema di unicita' del limite}
\subsubsection{teorema di permanenza del segno}
\subsubsection{teorema di confronto}
\subsubsection{algebra dei limiti e forme indeterminate}
\subsubsection{limite di funzioni monotone}
\subsubsection{limiti funzioni elementari}
\subsubsection{i limiti notevoli}
\begin{mydefinition}[limite notevole di nepero]
    
\end{mydefinition}
\subsubsection{Continuita' e discontinuita'}
tipi di continuita'
\begin{itemize}
    \item discontinuita' eliminabile
    \item discontinuita' a salto
    \item discontinuita' essenziale
\end{itemize}
\subsubsection{algebra delle funzioni continue}
\subsubsection{continuita' della composizione}
\subsubsection{proprieta' delle funzioni continue}
\begin{mydefinition}[teorema degli zeri]
    
\end{mydefinition}
\begin{mydefinition}[teorema dei valori intermedi]
    
\end{mydefinition}
\subsubsection{massimi e minimi assoluti}

\subsubsection{teorema di Weierstrass}

\subsubsection{il simpbolo di o piccolo}

\section{il calcolo differenziale}
\begin{mydefinition}[la derivata]
    
\end{mydefinition}
\begin{mydefinition}[significato geometrico]
    
\end{mydefinition}
\begin{mydefinition}[derivabilita' implica continuita']
    ma non tutte le funzioni continue sono derivabili
\end{mydefinition}
\subsection{derivata di funzioni elementari}
\subsection{Regole di derivazione }
\subsubsection{teorema di linearita'}
\subsubsection{teorema-regola di Liebnitz}
\subsubsection{teorema regola della catena}
\subsubsection{teorema derivazione della funzione inversa}
\subsubsection{corollario derivazione del quozionte}
\subsubsection{massimi e minimi relativi}
\subsubsection{teorema di Fermat}
\begin{mydefinition}[punti critici]
    
\end{mydefinition}
\subsubsection{teorema di Rolle}
\subsubsection{teorema di Lagrange}
\subsubsection{caratterizzazione delle funzioni monotone su intervalli}
\subsubsection{caratterizzazione delle funzioni costanti su intervalli}
\subsection{calcolo differenziale pt2}
\subsubsection{teorema di cauchy}
\subsubsection{teorema di De l'Hopital}
\subsubsection{criterio di derivabilita'}
\subsubsection{derivate successive}
\subsubsection{insiemi convessi e funzioni convesse}
\begin{mydefinition}[funzione convessa]
    
\end{mydefinition}
\begin{mydefinition}[convessita' e derivata prima]
    
\end{mydefinition}
\begin{mydefinition}[convessita' e derivata seconda]
    
\end{mydefinition}
\begin{mydefinition}[punti di flesso]
    
\end{mydefinition}
\subsubsection{Formula di Taylor}
\begin{mydefinition}[polinomio di Taylor]
    \begin{itemize}
        \item formula di taylor con il resto di peano
        \item formula di taylor con il resto di lagrange
    \end{itemize}
\end{mydefinition}

\section{integrali}
\begin{mydefinition}[integrale di Riemann]
    somme inferiori e superiori
\end{mydefinition}
\subsubsection{significato geometrico}
\subsubsection{criteri di integrabilita'}
\subsection{proprieta' degll'integrale}
\begin{itemize}
    \item Linearita'
    \item Additivita' rispetto al dominio
    \item positivita'
    \item monotonia
    \item $|\int_a^bf(x)dx|\leq \int_a^bf(x)dx$
\end{itemize}
\emph{N.B.}\begin{itemize}
    \item $\int_a^af(x)dx=0$
    \item se a<b $\int^a_bf(x)dx:=-\int^b_af(x)dx$
\end{itemize}
\begin{mydefinition}[il teorema della Media]
    
\end{mydefinition}
\subsection{la funzione integrale}
\subsubsection{il teorema del calcolo integrale}
\subsubsection{Primitiva}
\subsection{integrali indefiniti}
\subsection{formula fondamentale del calcolo integrale}
\subsubsection{calcolo degli integrali: integrazione per sostituzione}
\subsubsection{simmetrie negli integrali}
\subsubsection{calcolo degli integrali: integrazione per parti}
\subsubsection{calcolo degli integrali: integrazione delle funzioni razionali}
\begin{itemize}
    \item n>m
    \item n $\geq$ m
    \item  m=2
\end{itemize}
\section{serie numeriche e integrali generalizzati}
\subsection{serie numeriche}
\subsection{serie telescopiche}
\subsection{condizioni necessarie per la convergenza di una serie}
\subsection{alcune osservazioni sul carattere di una serie}
\subsection{serie a termini positivi}
\subsection{criterio del confronto}
\subsection{criterio del confronto asintotico}
\subsection{serie armonica generalizzata}
\subsection{criterio del rapporto e della radice n-esima}
\subsection{criterio di convergenza assoluta}
\subsection{criterio di leibnitz}
\subsection{integrali generalizzati}
\begin{itemize}
    \item integrale generalizzato in intervallo limitato
    \item integrale generalizzato in intervallo illimitato
\end{itemize}
\subsection{criterio del confronto}
\begin{itemize}
    \item criterio del confronto
    \item criterio del confronto asintotico
\end{itemize}
\subsection{criterio di convergenza assoluta}
\subsection{serie a integrali generalizzati}
\subsection{criterio integrale}
\section{Equazioni Differenziali Ordinarie}
\subsection{problema di cauchy}
\subsection{esistenza e unicita' locale di soluzioni}
\subsection{equazioni a variabili separabili}
\subsection{equazioni lineari del primo ordine}
\subsection{equazioni lineari del secondo ordine}
\section{Mega Riassunto Pazzo Della Morte Finale}
\subsection{Il calcolo differenziale}
\subsection{Integrali}
\subsection{Serie numeriche}
\subsection{integrale generalizzato}
\subsection{equazioni differenziali ordinarie}
\end{document}

